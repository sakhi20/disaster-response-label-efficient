% Main paper LaTeX file
% Compile with: pdflatex main.tex

\documentclass[10pt,twocolumn]{article}
\usepackage[margin=1in]{geometry}
\usepackage{graphicx}
\usepackage{amsmath}
\usepackage{booktabs}
\usepackage{hyperref}
\usepackage{cite}

\title{Disaster Response Building Damage Detection\\
       with Geospatial Foundation Models}

\author{Sakhi Patel \and Vivek Vanera \and Mulya Patel}

\date{\today}

\begin{document}
\maketitle

\begin{abstract}
We present a comparative study of geospatial foundation models and Vision Transformers
for post-disaster building damage detection using the xBD dataset. Our approach
fine-tunes NASA's Prithvi-100M model on bi-temporal satellite imagery to classify
building damage into four severity categories. We compare against standard ViT and
CNN baselines, demonstrating the value of geospatial pre-training for disaster response.
\end{abstract}

\section{Introduction}
% TODO

\section{Related Work}
% TODO

\section{Dataset}
% TODO: xBD description

\section{Methodology}
% TODO: Model architectures

\section{Experiments}
% TODO: Results tables

\section{Conclusion}
The proposed label-efficient framework leveraging geospatial foundation models demonstrates significant potential for rapid disaster response. By reducing the reliance on large-scale expert-labeled datasets, we enable faster deployment of automated damage assessment systems.

\section{Future Work}
Future research will focus on the integration of Synthetic Aperture Radar (SAR) imagery to provide all-weather monitoring capabilities. This extension, developed in collaboration with Prof. Mikhail Gilman (NCSU Mathematics), aims to fuse bi-temporal optical data with multi-view SAR geometry to improve detection robustness through clouds, smoke, and darkness.

\bibliographystyle{plain}
\bibliography{references}

\end{document}
